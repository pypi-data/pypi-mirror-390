% ACL Style Template - Main Document
% Two-column layout with footnote-based metadata
%
% IMPORTANT: ACL requires 11pt document class and manual \aclmetadata call
% Options: final (default), confidential, internal, draft, review
% See acl.sty and STYLE_CONTRACTS.md for details

\documentclass[11pt]{article}  % ACL requires 11pt for proper formatting

% Load ACL style with desired options
\usepackage[final]{acl}  % Options: final, confidential, internal, draft, review

% Standard packages (most are auto-loaded by acl.sty)
\usepackage{hyperref}
\usepackage{url}
\usepackage{booktabs}
\usepackage{amsfonts}
\usepackage{amsmath}
\usepackage{graphicx}

% ============================================================================
% DOCUMENT METADATA (Optional - uncomment and customize as needed)
% ============================================================================

% Organization information
% \setorganization{Your Organization}

% Document classification
% \setdoctype{Technical Analysis}
% \setdocid{DOC-2025-001}

% Note: Metadata appears as FOOTNOTE on first page ONLY if set
% Format: ¹ STATUS — ORGANIZATION — Document Type — DOC-ID
% IMPORTANT: You MUST add \aclmetadata to the \author{} command below!
% If no metadata is set, the footnote will only show status (if using confidential/internal/draft)

% ============================================================================
% DOCUMENT INFORMATION
% ============================================================================

\title{Document Title}

% IMPORTANT: Add \aclmetadata at the end of author field to display metadata footnote
\author{
  Author Name \\
  Department or Affiliation \\
  Organization \\
  \texttt{email@example.com}
  \aclmetadata  % <-- REQUIRED for metadata footnote!
}

% ============================================================================
% DOCUMENT CONTENT
% ============================================================================

\begin{document}

\maketitle

\begin{abstract}
This is the abstract of your document. Provide a concise summary of your work, including the main objectives, methodology, key findings, and conclusions. Keep it brief and informative. ACL uses a two-column layout with compact spacing.
\end{abstract}

\section{Introduction}

This is the introduction section. Provide background information, motivation, and context for your work. ACL style uses a two-column format optimized for conference papers, making it ideal for technical documents with dense content.

\subsection{Background}

Describe the background and related work relevant to your document.

\subsection{Objectives}

State the main objectives or goals of this work.

\section{Methodology}

Describe your approach, methods, or procedures. The two-column layout allows for efficient use of space while maintaining readability.

\section{Results}

Present your findings, results, or analysis.

\subsection{Key Findings}

Highlight the most important results. Tables and figures work well in the two-column format.

\begin{table}[h]
\centering
\caption{Example Results Table}
\label{tab:results}
\begin{tabular}{lcc}
\toprule
Method & Accuracy & Time \\
\midrule
Method A & 92.3\% & 1.2s \\
Method B & 95.1\% & 1.8s \\
\bottomrule
\end{tabular}
\end{table}

\section{Discussion}

Discuss the implications and significance of your results.

\section{Conclusion}

Summarize your main points and conclusions. The ACL style's compact format is ideal for technical reports, research summaries, and documentation that benefits from dense information presentation.

% References (if using bibliography)
% ACL provides special citation commands:
% \cite{} - same as \citep{} - gives "(Author Year)"
% \newcite{} - same as \citet{} - gives "Author (Year)"
% \shortcite{} - same as \citeyearpar{} - gives "(Year)"

% \bibliographystyle{acl_natbib}
% \bibliography{references}

\end{document}
